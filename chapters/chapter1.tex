
\chapter{Introduction}
\begin{refsection}
\section{Background of the Problem}

Today, the prevalence of fake news has become a widespread concern, with the widespread adoption of social media platforms exacerbating its dissemination.\cite{zhang_fake_2023}. The proliferation of fake news poses potential harm to both individuals and society at large. The prevalence of this issue has escalated rapidly in recent years.\cite{inproceedings1}. Many people are unable to compare fake news to non-fake news. Fake news can easily build a false positive or negative perception about a person, or an event. Not knowing fake news from non-fake news may lead to problems like; confusion and misinformation. People may become confused about what is true and what is not, leading to misinformation spreading rapidly. One good example of the effect of fake news is loss of trust, if there's no clear distinction between fake news and reliable news sources, trust in media and information sources can erode. This loss of "trust" can have far-reaching consequences for society, including reduced civic engagement and polarization. Also, people may struggle to make important decisions. People can easily be manipulated by spreading fake news. The worst effect may lead to social division. The inability to compare fake news to non-fake news can exacerbate social divisions as people become entrenched in their beliefs based on misinformation, leading to polarization and conflict within society.


Throughout history, false news has been abundant. It influenced significant events like the Enlightenment, where Voltaire reacted against the Catholic Church's misleading explanation of the 1755 Lisbon Earthquake. Even in the early days of American colonialism, fake stories, such as one about France's Louis XIV, circulated. Racist attitudes in the 1800s fueled fabricated tales about African Americans. Nazi propaganda also exploited false narratives to fuel anti-Semitic hatred. In the late 19th century, the rivalry between newspaper moguls Joseph Pulitzer and William Hearst led to sensationalized reporting, dubbed "yellow journalism," which contributed to the Spanish-American War. However, public outcry for more trustworthy news led to the establishment of outlets like "\textbf{The New York Times}" in the early 20th century. While yellow journalism waned for a time, the advent of online news saw its resurgence\cite{flanagin_online_2017}.

In the Philippines, fake news is a significant problem, with many encountering it in their daily lives. What's concerning is that a considerable number of Filipinos struggle to distinguish between fake and genuine news. Our nation contributes to the global phenomenon of disinformation, actively spreading fake news, often bolstered by political trolls. These trolls serve as both disseminators and sometimes creators of disinformation within the Philippine public sphere\cite{article1}. Many trolls are present in our country, a good example of this is political trolls. Trolling is a behavior characterized by being excessively negative online towards individuals or organizations, typically in a highly inflammatory or offensive manner\cite{political_trolling}. False information is a serious issue as it distorts the truth and fosters societal discord. As informed individuals, we need to identify and report false information. We must collaborate to educate those who are misinformed and embrace the truth. The victory in our war against disinformation should be measured by the number of individuals swayed by our fact-checking. We need a systematic strategy to combat trolls. The primary ethical issue with false information is not the false information itself, but our struggle against it.\cite{article1}


\section{Statement of the Problem}

The main purpose of this study is to combat the extensive spread of fake news within the Philippine news sector. The propagation of misinformation and disinformation not only leads to a significant divergence of individuals from reality but also poses severe threats to societal cohesion, democratic processes, and individual well-being. The dissemination of fake news has the potential to manipulate public opinion, distort factual understanding, and incite unnecessary fear or outrage among the populace. Moreover, the unchecked proliferation of falsehoods in media platforms can have ruinous consequences, including tarnishing reputations, undermining trust in institutions, and exacerbating social divisions. Given these multifaceted challenges, it is imperative to undertake comprehensive research to understand the mechanisms driving the dissemination of fake news, identify the vulnerable segments of society most susceptible to its influence, and develop effective strategies and interventions to mitigate its harmful effects. Failure to address this pressing issue not only threatens the integrity of the Philippine media landscape but also threatens the fundamental rights and freedoms of its Filipino citizens.


\section{Objectives of the Study}

\subsection{General Objective}

To combat Fake News by detecting Fake News in the Philippine News using Natural Language Processing(NLP) Algorithms.

\subsection{Specific Objectives}

More Specifically, this study aims to:

\begin{enumerate}
    \item Identify the relevant sources and collecting data for preprocessing. Model training and comparison of results.
    \item Optimize of the models with the best result for fake news detection.
    \item Identifying the model with the best result and creation of the web plugin.
\end{enumerate}


\section{Significance of the Study}

The following entities that will benefit from this study are the community, the researcher, and the future researchers.

{\bf Community}

With the help of this study, the community will be able to distinguish fake news from non-fake news and they can improve their living. They can avoid being scammed, easily manipulated, and other negative effects of spreading fake news.


{\bf Researcher}

Through this study the researchers will accomplish the requirements to pass the course, Computer Science Thesis 1, and also we can gain more knowledge about the growing trend of fake news and how to combat it using our proposed project.


{\bf Future Researchers}

To Future Researchers, the outcome of this study is beneficial for them as it may be used for future studies about the topic of Fake News Detection Using NLP.


\section{Scope and Limitation}

Detecting fake news in Philippine news can be a great contribution to the community as it helps to enlighten the minds of everyone on what is real and fake. In six months, we will develop this project that can detect fake news in the Philippine news. 


The definition and identification of fake news can vary depending on the context, the source, and the audience. There is no clear and universal criterion for what constitutes fake news, and different stakeholders may have different perspectives and opinions on the truthfulness and credibility of a news article. The data collection and labeling process can be costly, time-consuming, and prone to errors and biases. It can be difficult to find and access reliable and diverse sources of fake news articles and to verify and annotate them with high-quality labels. Moreover, the data distribution and characteristics can change over time, as new types of fake news emerge and evolve. The fake news detection models can face technical and ethical challenges, such as the lack of interpretability and explainability, the vulnerability to adversarial attacks and countermeasures, and the potential impact on the freedom of speech and the privacy of the users. The models need to be robust, transparent, and accountable, and to respect the rights and values of the stakeholders involved.

\section{Project Dictionary}

The Project Dictionary contains the technical terms that defined the concept and operation of this study:

\begin{itemize}
    \item \textbf{Trust} to rely on the truthfulness or accuracy of.\cite{trust}
    \item \textbf{Fake news} is false information, intentionally created and spread to appear as real news, aiming to deceive readers. It refers to a specific false piece of information, not a type of news source or individual.\cite{cunningham_fake-news}
    \item \textbf{Natural Language Processing (NLP).} is a subset of AI, that allows computers to understand and manipulate human language. It’s used in voice or text interactions, often unnoticed by users. NLP powers virtual assistants like Siri and Alexa, enabling them to comprehend and respond in human language. It works with both text and speech across all languages. NLP is also behind web searches, spam filters, automatic translations, document summaries, sentiment analysis, and grammar checks. Some email services even use NLP to suggest responses based on message content.\cite{nlp}
    \item \textbf{Web Scraping} Web scraping is a method where bots extract data from websites. It’s used in various digital businesses for data collection.\cite{web_scraping}
\end{itemize}

%=======================================================%
%%%%% Do not delete this part %%%%%%
\clearpage

\printbibliography[heading=subbibintoc, title={\centering Notes}]
\end{refsection}